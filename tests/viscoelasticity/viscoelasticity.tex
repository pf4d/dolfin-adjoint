\documentclass{amsart}
\usepackage{amsmath}
\usepackage{amssymb}

\DeclareMathOperator{\Div}{div}
\DeclareMathOperator{\Grad}{grad}
\DeclareMathOperator{\Skew}{skew}
\newcommand{\avg}[1]{[#1]}
\newcommand{\inner}[2]{\langle #1, #2 \rangle}

\begin{document}

\title{Quasi-static Viscoelasticity}
\maketitle

\section{The basics}

System of equations: find $\sigma_0, \sigma_1, v$ such that
\begin{align}
  A_E^{j} \dot \sigma_j + A_{V}^{j} \sigma_j - \varepsilon(v) &= 0
  \quad j = 0, \cdots, n-1 \\
  \sigma &= \sum_{j} \sigma_j \\
  \Div \sigma & = f \\
  \Skew \sigma &= 0
\end{align}
with initial conditions:
\begin{equation}
  \sigma_j(0) = \sigma_j^0
\end{equation}
for all $j$ where $A_E^{j} \not = 0$, and boundary conditions
\begin{align}
  \sigma \cdot n &= g \quad \text{ on } \partial \Omega_N \\
  v &= v_{D} \quad \text{ on } \partial \Omega_D
\end{align}

First, note that
\begin{equation}
  \varepsilon(v) = \frac{1}{2} (\Grad(v) + \Grad(v)^T)
  = \Grad(v) + \frac{1}{2}(\Grad(v)^T - \Grad(v))
\end{equation}
Introduce the variable $\gamma$:
\begin{equation}
  \gamma = \frac{1}{2}(\Grad(v) - \Grad(v)^T)
\end{equation}
which is clearly skew-symmetric, and moreover
\begin{equation}
  \varepsilon(v) = \Grad(v) - \gamma
\end{equation}

Rewriting the system, we have:
\begin{align}
  A_E^{j} \dot \sigma_j + A_{V}^{j} \sigma_j - \Grad(v) + \gamma &= 0
  \quad j = 0, 1 \\
  \sigma &= \sum_{j} \sigma_j \\
  \Div \sigma & = f \\
  \Skew \sigma &= 0
\end{align}

\section{Discretizing in space}

Take $\tau_j$, $w$ and $\eta$ test functions in space, multiply, and
integrate
\begin{align}
  \inner{A_E^{j} \dot \sigma_j}{\tau_j} + \inner{A_{V}^{j} \sigma_j}{\tau_j} - \inner{\Grad(v)}{\tau_j} + \inner{\gamma}{\tau_j} &= 0
  \quad j = 0, 1 \\
  \inner{\Div \sigma}{w} & = \inner{f}{w} \\
  \inner{\Skew \sigma}{\eta} &= 0
\end{align}
with $\sigma = \sum_j \sigma_j$. And integrating by parts getting rid of the $\Grad(v)$'s:
\begin{align}
  \inner{A_E^{j} \dot \sigma_j}{\tau_j} + \inner{A_{V}^{j} \sigma_j}{\tau_j} + \inner{v}{\Div \tau_j} + \inner{\gamma}{\tau_j}  &= \inner{v}{\tau_j \cdot n}_{\partial \Omega}
  \quad j = 0, 1 \\
  \inner{\Div \sigma}{w} & = \inner{f}{w} \\
  \inner{\Skew \sigma}{\eta} &= 0
\end{align}
Since $\gamma$ is skew, we can clean up the structure:
\begin{align}
  \inner{A_E^{j} \dot \sigma_j}{\tau_j} + \inner{A_{V}^{j} \sigma_j}{\tau_j} + \inner{\Div \tau_j}{v} + \inner{\Skew \tau_j}{\gamma}  &= \inner{v}{\tau_j \cdot n}_{\partial \Omega}
  \quad j = 0, 1 \\
  \inner{\Div \sigma}{w} + \inner{\Skew \sigma}{\eta} & = \inner{f}{w}
\end{align}
for all $\tau_j$, $w$ and $\eta$

\section{Discretizing in time}

We want to apply a TR-BDF$_2$ scheme; in other words, a two-step
temporal discretization with first a trapezoidal rule step and then a
BDF$_2$ step.

The algorithm can then be summarized as:
\begin{enumerate}
\item
  Assume that $\sigma_j^{n}$ for the required $j$ is known. (Actually that all variables are known...)
\item
  TR-step: Let $k_n$ denote the timestep
  \begin{align}
    \inner{A_E^{j} k_n^{-1} (\sigma_j^{\bigstar} - \sigma_j^n)}{\tau_j}
    + \inner{A_{V}^{j} \avg{\sigma_j^{\bigstar} }}{\tau_j}
    + \inner{\Div \tau_j}{\avg{v^{\bigstar}} }
    + \inner{\Skew \tau_j}{\avg{\gamma^{\bigstar}}}
    &= \inner{\avg{v^{\bigstar}}}{\tau_j \cdot n}_{\partial \Omega}
    \quad j = 0, 1 \\
    \inner{\Div \avg{\sigma^{\bigstar}}}{w} + \inner{\Skew \avg{\sigma^{\bigstar}}}{\eta} & = \inner{\avg{f}}{w}
  \end{align}
  where $2 \avg{v^{\bigstar}} = v^{\bigstar} - v^{n}$
\end{enumerate}


\end{document}
